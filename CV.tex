\documentclass[10pt]{article}
\usepackage{array, xcolor, lipsum, bibentry}
%\usepackage[margin=3cm]{geometry}
\usepackage[left=3cm,right=2cm,top=2.5cm,bottom=3cm]{geometry}  

%\title{\bfseries\huge Martin Vondrák}
%\author{vondrakmar@gmail.com, +420737778448}
\date{}
 
\definecolor{lightgray}{gray}{0.8}
\newcolumntype{L}{>{\raggedleft}p{0.12\textwidth}}
\newcolumntype{R}{p{0.8\textwidth}}
\newcommand\VRule{\color{lightgray}\vrule width 0.5pt}
 
 
\begin{document}
%\maketitle
\begin{center}
\textbf{\huge{Martin Vondrák}}
\end{center}
\noindent Date of Birth: 29. 7. 1996
\newline\newline
\noindent Email: vondrak@fhi-berlin.mpg.de
\newline\newline
\noindent Phone number: +49 160 2659298 %+420737778448

\section*{Current research interest}
\begin{itemize}
	\setlength\itemsep{0em}
	\item Machine learning electrostatic interactions in materials % Development of machine learned electrostatic potentials

\end{itemize}

\subsection*{Past Research interests}
\begin{itemize}
	\setlength\itemsep{0em}
	\item Molecular mechanics simulations of graphene
	\item Interactions of small RNA hairpins with graphene
	\item Molecular dynamics simulations of graphene based supercapacitors
\end{itemize}



\section*{Work Experience}
\begin{tabular}{L!{\VRule}R}
2021--now& {\bf Ph.D. candidate position at the Fritz Haber Institut MPG and the University of Bayreuth, Germany}\\[3pt]
2015--2021& Science center Fort Science\\[3pt]
&I worked as an edutainer in math, physics and chemistry exhibitions.\\

\end{tabular}
 
\section*{Education}
\begin{tabular}{L!{\VRule}R}
2024--now& Ph.D. candidate at the University of Bayreuth, Germany. Supervised by Dr. Johannes Margraf, and Prof. Karsten Reuter\\[5pt]
10.-12. 2023& Visiting student in the group of Professor Gábor Csányi, Department of Engineering Engineering of Cambridge University, Cambridge, United Kingdom\\[5pt]
2021--2023& Ph.D. candidate at the Fritz Haber Institut MPG, Germany. Supervised by Dr. Johannes Margraf, and Prof. Karsten Reuter\\[5pt]
2019--2021& MSc in Physical Chemistry, Palacký University, Czech Republic\\[3pt]

& Final exams with marks (the scale of grades from A (best) to F (worst)): \newline
Modeling of biostructures and bioinformatics~(\textbf{A}), Quantum Chemistry and Chemical Structure (\textbf{A}),  Analytical Chemistry~(\textbf{A}), Physical Chemistry~(\textbf{A}), Final MSc Thesis Defense~(\textbf{A})\\[3pt]
2019--2021&Qualification Course for Teachers of Chemistry, Palacký University, CR\\[5pt]
2016--2019&BSc in Chemistry, Palacký University, Czech Republic\\[3pt]
& Final exams with marks (the scale of grades from A (best) to F (worst)): \newline
General and Inorganic Chemistry~(\textbf{A}), Organic Chemistry (\textbf{A}),  Analytical Chemistry~(\textbf{B}), Physical Chemistry~(\textbf{A}), Final BC Thesis Defense~(\textbf{A})
\end{tabular}
 
%\section*{Languages}
%\begin{tabular}{L!{\VRule}R}
%Klingon&Mother tongue\\
%{\bf English}&{\bf Fluent}\\
%French&Fluent (DELF 2010)\\
%Japanese&Fair\\
%\end{tabular}
 
 
\section*{Publications and Awards}
\begin{tabular}{L!{\VRule}R}
2023 & First author of Vondrák, M. et al. q-pac: A Python package for machine learned charge equilibration models. J. Chem. Phys. 159, 054109 (2023)\\[2.2em]

2022 & Author of Pykal, M. et al. Accessibility of Grafted Functional Groups Limits Reactivity of Covalent Graphene Derivatives.  Appl. Surf. Sci. 598, (2022)\\[2.2em]

2018 & Coauthor of Li, Q. et al. RNA nanopatterning on graphene. 2D Mater. 5, (2018)\\[1.2em]

2016 & Students' Professional Activities, 3rd place in Chemistry category\\[3pt]
&Name of the project: Structural Changes Accompanying the Process of Graphene Oxidation\\[2.2em]

2015 & Czech Little Head Genus 2015 Award\\[1.2em]

2015 & The Learned Society Award for Students' Professional Activities\\[1.2em]

2015 & Students' Professional Activities, 2nd shared (50 \%) place in Chemistry category\\[3pt]
& Name of the project: The Catalysis of the Nucleolytic Ribozymes\\[1.2em]

2015 & Students' Professional Activities, 4th place in Math category\\[3pt]
&Name of the project: Statistical Analysis of Composition Tables in Coordinates\\[1.2em]

2014 & Students' Professional Activities, 5th place in Math category\\[3pt]
&Name of the project: Statistic Analysis of Independence in Four-field Data Tables 
\end{tabular}

\section*{Other Activities}
\begin{tabular}{L!{\VRule}R}
2019&Foundation of Reproducibilitea in Olomouc, Czech Republic\\[3pt]
&Czech branch of global journal club started 2016 at University of Oxford initiative focused on open and reproducible science, founded and organized by Chemistry Club UP \\[3.2em]

2017&Foundation of Chemistry club UP\\[3pt]
&Student organization at Palacký University holding lectures from chemistry given by experts, popularizing chemistry for high school students and organizing excursions in chemistry laboratories\\[3.2em]

2016&Participation on the international conference MILSET Expo-Sciences Europe 2016 \\[3pt]
&Presentation of the project The Catalysis of the Nucleolytic Ribozymes at the MILSET Expo-Sciences Europe 2016 in Toulouse, France\\[2.2em]
2015&World Science Conference -- Israel \\[3pt]
&Workshops and lectures from Nobel laureates, research-based Israeli companies, and researchers from Hebrew University of Jerusalem.\\[5pt]
\end{tabular}

% \section*{Skills}
% \begin{itemize}
% %	\setlength\itemsep{0em}
% 	\item Active user of Linux
% 	\item Active user of Python and Latex  
% 	\item Passive knowledge of C and C$\#$ 
% 	\item English Language - C1 level in CAE exam
% \end{itemize}

\end{document}